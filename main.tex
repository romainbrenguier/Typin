\documentclass{article}
\usepackage{amsmath,amsfonts,amssymb}
\newcommand{\substitution}[2]{\ensuremath{#1_{|#2}}}

\title{Types with Integer Parameters}
\date{}

\begin{document}

\maketitle{}

Types are a useful abstraction of programs, which allow detection of errors at compilation time.
However they often lack expressive power.
For example knowing the size of an array at compilation time could prevent out of bound accesses.
We propose to add parameters to type systems used in functional programs to be able to express properties like the size of an array directly in the types.

\section{A simple language}
We consider a simple functional programming language in which the terms are constructed from the following grammar:
\[ t ::= x \mid \texttt{fun}\ x \rightarrow t \mid t~t \mid \texttt{let}\ x = t\ \texttt{in}\ t \]
The name $x$ designate a variable name, the second rule is abstraction (i.e. function definition), the third function application and the last one is a variable declaration.


\section{A Simple typing rule}
We augment the classical typing rules with integer intervals.
Types are generated by the grammar:
\[ \tau ::= [a,b] \mid \tau \rightarrow \tau \]
Where $a$ and $b$ are in $\mathbb{N} \cup \{ -\infty , +\infty \}$ and define an interval.
To simplify notations we will write $[a]$ for the interval $[a,a]$, $[\le a]$ for $[-\infty,a]$, and $[\ge a]$ for $[a, +\infty]$.


The only rule we change is that to apply a function, the interval on the left of the arrow must contain the interval of the argument:

\[ \frac{f : [a,b] \rightarrow \tau \hspace{1cm} x : [c,d]}{ f~x : \tau} ~\text{if } a \le c \le d \le b\]

We this new rule we can for instance ask that the argument of a function should be positive.
However this is not very precise.
For instance one would like to say that the result of $a~\texttt{mod}~b$ is in the interval $[0,b]$.
For this the type of the result must depend on the type of the arguments so we need to add variables inside the intervals.

\section{Adding variables in ponctual type}

We now add the possibility to have variables inside types.
But for now we restrict to ponctual intervals.
The new grammar is
\begin{align*}
 \tau & ::= [\alpha] \mid \tau \rightarrow \tau \\
 \alpha & ::= i \mid x + i 
\end{align*}
Where $i$ is an integer and $x$ a variable symbol.
We add the following new typing rule for function application when variables are involved:
\[ \frac{t : [x+a]}{ x:b \models t : [a+b]} \]

This addition can be useful for hardware design: in pipelined architercture in general we need to avoid variables that are at different level of the pipeline.
We can make so that the type system take into account the level at which we are and forbid using different levels unless we explicitly mention it:
\begin{align*}
\texttt{and\_gate} & : [x] \rightarrow ([x] \rightarrow [x]) \\
\texttt{register} & : [x] \rightarrow [x+1] \\
\texttt{cast} & : [x] \rightarrow [y]
\end{align*}


\section{Variables in intervals}

We now add the possibility to have variables inside types.
The new grammar is
\begin{align*}
 \tau & ::= [\alpha,\alpha] \mid \tau \rightarrow \tau \\
 \alpha & ::= i \mid +\infty \mid -\infty \mid x + i 
\end{align*}
Where $i$ is an integer and $x$ a variable symbol.
We add the following new typing rule for function application when variables are involved:
%\[ \frac{f : [x+a,b] \rightarrow \tau \hspace{1cm} t : [c,d]}{ f~t : \tau[x\leftarrow c-a]} ~\text{if } d \le b\]
\[ \frac{f : [x+a,b] \rightarrow \tau \hspace{1cm} t : [\alpha,d]}{ f~t : \substitution{\tau}{x\leftarrow \alpha-a}} ~\text{if } d \le b
\hspace{1cm} \frac{f : [a,x+b] \rightarrow \tau \hspace{1cm} t : [c,\alpha]}{ f~t : \substitution{\tau}{x\leftarrow \alpha-b}} ~\text{if } a \le c\]

\[ \frac{f : [x+a,y+b] \rightarrow \tau \hspace{1cm} t : [\alpha,\beta]}{ f~t : \substitution{\tau}{x\leftarrow \alpha-a, y\leftarrow \beta-b}} 
\]
 

This allows for instance to precisely describes addition by a constant or remainder in euclidian devision:
\begin{align*}
\texttt{add10} & : [x] \rightarrow [x+10] \\
\texttt{mod} & : [\ge 0] \rightarrow ( [x] \rightarrow [0,x] )
\end{align*}

\section{Expression with several variables}
The type system we have described so far does not allow us to precisely characterize addition because we would need to sum two variables:
\[\texttt{add} : [x] \rightarrow ( [y] \rightarrow [x+y]) \]



\end{document}
